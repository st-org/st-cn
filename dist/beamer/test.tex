\documentclass{beamer}
\usetheme{Madrid}

\def\Re#1{\operatorname{Re}\left(#1\right)}

\title[No Largest Prime]{There Is No Largest Prime Number}
\subtitle{With an introduction to a new proof technique}

\author[Euclid, DDU]{
    Euclid of Alexandria\inst{1}
    \and
    DDU\inst{2}
}
\institute[Alexandria, ST]{
    \inst{1}
    \texttt{euclid@alexandria.edu}\\
    Department of Mathematics\\
    University of Alexandria
    \and
    \inst{2}
    \texttt{ddu6@protonmail.com}\\
    Simple Text Orgnazition
}
\date[ISPN '80]{27th International Symposium of Prime Numbers, --280}

\begin{document}

\frame{\titlepage}

\begin{frame}
    \frametitle{Outline}
    \tableofcontents[pausesections]
\end{frame}

\AtBeginSection[]
{
  \begin{frame}
    \frametitle{Outline}
    \tableofcontents[currentsection]
  \end{frame}
}

\section{Motivation}

\subsection{What Are Prime Numbers?}

\begin{frame}
    \frametitle{What Are Prime Numbers?}

    \begin{definition}
        A \alert{prime number} is a number that has exactly two divisors.
    \end{definition}

    \begin{example}
        \begin{itemize}
            \item 2 is prime (two divisors: 1 and 2).
            \pause
            \item 3 is prime (two divisors: 1 and 3).
            \pause
            \item 4 is not prime (\alert{three} divisors: 1, 2, and 4).
        \end{itemize}
    \end{example}
\end{frame}

\section{Results}

\subsection{Proof of the Main Theorem}

\begin{frame}
    \frametitle{Proof of the Main Theorem}
    \framesubtitle{The proof uses \textit{reductio ad absurdum}}

    \begin{theorem}
        There is no largest prime number.
    \end{theorem}

    \begin{proof}
        \begin{enumerate}
            \item<1-| alert@1> Suppose $p$ were the largest prime number.
            \item<2-> Let $q$ be the product of the first $p$ numbers.
            \item<3-> Then $q + 1$ is not divisible by any of them.
            \item<1-> But $q + 1$ is \only<4->{greater than $1$, thus }divisible by some prime number not in the first $p$ numbers.\qedhere
        \end{enumerate}
    \end{proof}

    \uncover<5->{The proof used \textit{reductio ad absurdum}.}
\end{frame}

\section{What's More}

\subsection{An Open Question}

\begin{frame}
    \frametitle{An Open Question}

    \begin{columns}
        \column{.45\textwidth}
        \begin{block}{Answered Questions}
            How many primes are there?
        \end{block}

        \column{.45\textwidth}
        \begin{block}{Open Questions}
            Is every even number the sum of two primes?
            \cite{Goldbach1742}
        \end{block}
    \end{columns}

    \begin{thebibliography}{10}
        \bibitem{Goldbach1742}[Goldbach, 1742]
        Christian Goldbach.
        \newblock A problem we should try to solve before the ISPN '43 deadline,
        \newblock \emph{Letter to Leonhard Euler}, 1742.
    \end{thebibliography}
\end{frame}

\subsection{Another Open Question}

\begin{frame}
    \frametitle{Another Open Question}

    In complex plane, define $\zeta(s)=\sum\limits_{n=1}^\infty\frac1{n^s}$. Then, are all zeros of $\zeta$ in the strip $0\leq\Re{s}\leq1$ lie on the line $\Re{s}=\frac12$?
    \pause

    \begin{block}{Some Observation}
        If $\Re{s}>1$\pause, then
        \begin{align*}
            \zeta(s)&=\sum_{n=1}^\infty\frac1{n^s}\\
            &=\prod_p\frac1{1-p^{-s}}.
        \end{align*}
        \pause
       Thus $\zeta$ does not vanish when $\Re{s}>1$.
    \end{block}
\end{frame}

\subsection{An Algorithm For Finding Primes Numbers}

\begin{frame}[fragile]
    \frametitle{An Algorithm For Finding Primes Numbers}

    \begin{semiverbatim}
        \uncover<1->{\alert<0>{int main (void)}}
        \uncover<1->{\alert<0>{\{}}
        \uncover<1->{\alert<1>{ \alert<4>{std::}vector<bool> is_prime (100, true);}}
        \uncover<1->{\alert<1>{ for (int i = 2; i < 100; i++)}}
        \uncover<2->{\alert<2>{ if (is_prime[i])}}
        \uncover<2->{\alert<0>{ \{}}
        \uncover<3->{\alert<3>{ \alert<4>{std::}cout << i << " ";}}
        \uncover<3->{\alert<3>{ for (int j = i; j < 100;}}
        \uncover<3->{\alert<3>{ is_prime [j] = false, j+=i);}}
        \uncover<2->{\alert<0>{ \}}}
        \uncover<1->{\alert<0>{ return 0;}}
        \uncover<1->{\alert<0>{\}}}
    \end{semiverbatim}

    \visible<4->{Note the use of \alert{\texttt{std::}}.}
\end{frame}

\end{document}